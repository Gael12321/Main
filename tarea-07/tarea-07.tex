\documentclass{article}

\author{Gael Balderrama Dominguez}
\title{Tarea-07}
\date{Noviembre 2023}

\begin{document}
\begin{center}
    Gael Balderrama Dominguez \linebreak
    Tarea-07 \linebreak
    Noviembre 2023 \linebreak
\end{center}

1.-Extiende el lenguaje agregando un nuevo operador $minu$ que toma como argumento $n$ y regresa $-n$.Por ejemplo, el valor de minus(-(minus(5),9)) debe ser 14 \newline
Respuesta: \newline

\textbf{Especificacion Lexica :} \newline

Minus = Minus \\

\textbf{Especificacion Sintactica :} \newline

Concreta :Expression $\Rightarrow $ minus (Expression) \\

Abstracta : (minus-exp exp1)\\


\textbf{Especificacion Semantica :} \newline
$$
\frac{\mathcal{E} }{\mathcal{E} 
}
$$


2.-Extiende el lenguaje agregando operadores para la suma, multiplicacion y conciente de enteros. \newline
Respuesta: \newline

3.- Agrega un predicado de igualdad numerica $equal?$ y predicados de orden $greater?$ y $less?$ al conjunto de operaciones de lenguajes LET.
Respuesta: \newline

4.- Agrega opreaciones de procesamiento de listas al lenguaje, incluyendo $cons$,$car$,$cdr$,$null?$ y $emptylist$. Una lista debe poder contener cualquier valor expresado, incluyendo otra lista
Respuesta: \newline

5.- Agrega una operacion $list$ al lenguaje. Esta operacion debe tomar cualquier cantidad de argumentos y regresar un valor expresado de la lista de sus valores.\newline
Respuesta: \newline

7.-Incorpora al lenguaje expresiones $cond$. Usa la gramatica 
$$ Expression \Rightarrow cond {Expression \Rightarrow Expression}* end$$
En esta expresión, las expresiones de los lados izquierdos de los $\Rightarrow$  son evaluadas en orden
hasta que una de ellas regresa un valor verdadero. Entonces el valor de toda la expresión es
el valor de la expresión correspondiente al lado derecho de esa $\Rightarrow$ . Si ninguno de los lados
izquierdos es verdadero, la expresión debe reportar un error. \newline
Respuesta: \newline

8.- Cambia los valores del lenguaje para que los enteros sean los unicos valores expresados. Modifica if para que le valor de 0 sea tratado como falso y todos los otros sean tratados como verdaders. Modifica los predicados de manera consistente.\newline
Respuesta: \newline

9.-Como una alternativa al ejercicio anterior, agrega una nueva categoria sintáctica $Bool-exp$ de expresiones booleanas al lenguaje. Cambia la producción para expresiones condicionales para que sea \newline
$$
Expression \Rightarrow if Bool-exp  \: then \: Expression \: else  \:Expression
$$

Escribe Producciones apropiadas para $Bool-exp$ y especifica su semantica con $Value-of-bool-exp$ (Puedes abreviarlo como $B$). ¿En dónde terminan estando los predicados del ejercico 3 con este cambio? \newline

Respuesta: \newline

10.-Modifica la implementacion del intérprete agragando una nueva operacion $print$ que tome un argumento, lo imprime y regresa el entero 1. ¿Por qué esta operacion no es expresable en nuestro metodo de especificacion formal? \newline
Respuesta: \newline

11.-Extiende el lenguaje para que las expresiones $let$ puedan vincular una cantidad arbitraria de variables, usando la produccion,\newline
$$
Expression \Rightarrow  let \: \{ identifier = Expression* \}   \: in \: Expression
$$
Respuesta: \newline

12.- Extiende el lenguaje con unan expression Let* que funciona como en racket.\newline
Respuesta: \newline

13.- Agrega una expresion al lenguaje de acuerdo a la siguiente regla\newline

$$
Expression \Rightarrow  unpack \:\: \:  \{identifier\}* = Expression \:\:  in \: \:Expression
$$

tal que unpack x y z = lst in ... vincula x , y y z a los elementos de lst si lst es una lista con exactamente tres elementos, reportando un error en otro caso.\newline
Respuesta: \newline
\end{document}