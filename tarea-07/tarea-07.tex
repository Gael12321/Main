\documentclass{article}

\author{Gael Balderrama Dominguez}
\title{Tarea-07}
\date{Noviembre 2023}

\begin{document}
\begin{center}
    Gael Balderrama Dominguez \linebreak
    Tarea-07 \linebreak
    Noviembre 2023 \linebreak
\end{center}
1.-Extiende el lenguaje agregando un nuevo operador $minu$ que toma como argumento $n$ y regresa $-n$.Por ejemplo, el valor de minus(-(minus(5),9)) debe ser 14 \newline
Respuesta: \newline

2.-Extiende el lenguaje agregando operadores para la suma, multiplicacion y conciente de enteros. \newline
Respuesta: \newline

3.- Agrega un predicado de igualdad numerica $equal?$ y predicados de orden $greater?$ y $less?$ al conjunto de operaciones de lenguajes LET.
Respuesta: \newline

4.- Agrega opreaciones de procesamiento de listas al lenguaje, incluyendo $cons$,$car$,$cdr$,$null?$ y $emptylist$. Una lista debe poder contener cualquier valor expresado, incluyendo otra lista
Respuesta: \newline

5.- Agrega una operacion $list$ al lenguaje. Esta operacion debe tomar cualquier cantidad de argumentos y regresar un valor expresado de la lista de sus valores.\newline
Respuesta: \newline

7.-Incorpora al lenguaje expresiones $cond$. Usa la gramatica 
$$ Expression \rightarrow cond {Expression \Rightarrow Expression}* end$$
En esta expresión, las expresiones de los lados izquierdos de los => son evaluadas en orden
hasta que una de ellas regresa un valor verdadero. Entonces el valor de toda la expresión es
el valor de la expresión correspondiente al lado derecho de esa =>. Si ninguno de los lados
izquierdos es verdadero, la expresión debe reportar un error.
Respuesta: \newline



\end{document}