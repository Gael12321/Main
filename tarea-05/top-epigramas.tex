\documentclass{article}
\usepackage{graphicx} % Required for inserting images

\title{top-epigramas}
\author{a221206708 }
\date{September 2023}

\begin{document}

\maketitle

\section{Top 10 mejores epigramas}

\begin{itemize}
    \item 53. So many good ideas are never heard from gain once they embark in a voyage of the semantic gulf.
        \begin{itemize}
            \item \textbf{ Qué significa para ti el epigrama} 
                \newline Este epigrama significa que la mayor parte de ideas no se logran exponer a demas gente tanto por las habilidades de expresarse del autor o de las limitantes del idioma elegido para expresar
            \item \textbf{ Por qué lo elegiste}
                \newline Elegi este epigrama porque la idea de que el lenguaje limita de mayor o menor manera las expresion de ideas o pensamientos que podrian llegar a ser algo que cambiara el mundo me parecio interesante.
        \end{itemize}
    \item 102. One can’t proceed from the informal to the formal-by-formal means.
            \begin{itemize}
            \item \textbf{ Qué significa para ti el epigrama} 
            \newline Que el metodo de aprendizaje de uno afecta en la forma que uno utiliza efectivamente las diferentes fuentes de informacion, por ejemplo una persona autodidacta batallara a la hora de tener un aprendizaje estandarizado.
                
            \item \textbf{ Por qué lo elegiste}
            \newline Me gusto el como explica que el conocimiento puede venir de diferentes partes, no es necesario confiar completamente en los métodos académicos convencionales a la hora de aprender habilidades o conocimientos
            
        \end{itemize}
    \item 59. In English every word can be verbed. Would that it was so in our programming languages.
            \begin{itemize}
            \item \textbf{ Qué significa para ti el epigrama} 
            \newline El idioma inglés o cuál otro idioma usado por los humanos, tiene una flexibilidad innata a la hora de ser utilizado para expresar ideas o conceptos, por el contrario, los lenguajes de programación tienen una rigidez y una estructura con funciones y significados específicos, limitando la flexibilidad y creatividad 
                
            \item \textbf{ Por qué lo elegiste}
            \newline Me gusto la crítica que hace a los lenguajes de programación por su falta de creatividad en comparación con lenguajes normales 

        \end{itemize}
    \item 62. In computing, invariants are ephemeral.
            \begin{itemize}
            \item \textbf{ Qué significa para ti el epigrama} 
                \newline Las invariantes son utilizadas de manera muy común en ciclos durante un programa, haciendo que estas sean olvidadas o dejen de tomar importancia durante la ejecución mas grande del mismo, haciendo efímero el efecto de las mismas.
            
                
            \item \textbf{ Por qué lo elegiste}
            \newline El uso de las invariables es algo basico y fundamental a la hora de computacion y me gusto que el epigrama definiera su importancia como algo efimero, que es pasajero despues del contexto donde es utilizado
        \end{itemize}
    \item 3. Syntactic sugar causes cancer of the semicolon
            \begin{itemize}
            \item \textbf{ Qué significa para ti el epigrama} 
            \newline El agregar ayudas o como se definen aqui "Azucar sintactico" puede llegar a ser perjudicial a la hora de tener un programa mas claro y entendible, llegando a tener consecuencias negativas a lo largo plazo aunque puede llegar a ser lo contrario en un inicio
                
            \item \textbf{ Por qué lo elegiste}
            \newline Me gusto la forma en que esta escrito haciendo referencia a la enfermedad de cancer de colon pero utilizando conceptos de computacion como el punto y coma. 


            
        \end{itemize}
    \item 25. One can only display complex information in the mind. Like seeing, movement or flow or alteration of view is more important than the static picture, no matter how lovely.
            \begin{itemize}
            \item \textbf{ Qué significa para ti el epigrama} 
                \newline No importa que tan bonita sea la forma en la que mostremos la información de un programa, debemos de poder llegar a utilizar formas más intuitivas de demostrar esa información para que sea entendible de manera general
            
                
            \item \textbf{ Por qué lo elegiste}
            \newline Me gusta que expresa el uso de la mente humana a la hora de comprender informacion compleja de manera mas dinamica atravez del movimiento y de diferentes formas de presentar esa informacion
            
        \end{itemize}
    \item 26. There will always be things we wish to say in our programs that in all known languages can only be said poorly
            \begin{itemize}
            \item \textbf{ Qué significa para ti el epigrama} 
                \newline Los lenguajes de programación tienen una estructura más ordenada y jerárquica que el lenguaje normal, así dificultando el expresar ideas simples o complejos dentro de los programas. 
            
                
            \item \textbf{ Por qué lo elegiste}
            \newline La idea de poder llegar a expresar ideas de manera efectiva a programas me parece una dificultad a la hora de programar diferentes problemas. 
        \end{itemize}
    \item 59. Re graphics: A picture is worth 10K words - but only those to describe the picture. Hardly any sets of 10K words can be adequately described with pictures.
            \begin{itemize}
            \item \textbf{ Qué significa para ti el epigrama} 
            \newline Una imagen puede llegar a expresar más que las palabras, pero tienen que ser imágenes relacionadas con el ámbito en cuestión, se debe poder elegir cuando representar ideas complejas, atreves de imágenes y cuando atreves de texto.
                
            
                
            \item \textbf{ Por qué lo elegiste}
            \newline Las imágenes tienen dificultad a la hora de representar ideas abstractas como lo son diferentes, complejos e ideas a la hora de programar algo.
        \end{itemize}
    \item 48. The best book on programming for the layman is ”Alice in Wonderland”; but that’s because it’s the best book on anything for the layman.
            \begin{itemize}
            \item \textbf{ Qué significa para ti el epigrama} 
                \newline El uso de un libro no relacionado con programación para explicar que las ideas de programación y su lógica pueden ser vistos en ámbitos completamente separados. 
            
                
            \item \textbf{ Por qué lo elegiste}
            \newline Me gusto el uso de un libro de cultura general para ilustrar la creatividad y lógica que es necesaria para la programación. 
        \end{itemize}
    \item 108. Whenever two programmers meet to criticize their programs, both are silent.\
            \begin{itemize}
            \item \textbf{ Qué significa para ti el epigrama} 
            \newline El hecho de criticar el código de otra persona en el ámbito de programación puede llegar a ser difícil sin caer en la autocrítica o en el análisis exhaustivo del programa, se puede llegar a criticar una forma de realizar algo, aunque sea la manera óptima de hacerlo nomas porque el que anda analizando el código no logro llegar a la misma conclusión
                
            
                
            \item \textbf{ Por qué lo elegiste}
                \newline Me gusto el tema de la autocrítica que se da a entender el contexto del epigrama, también el trabajo de comprender un problema con solo leer la respuesta dada por otra persona.

            
        \end{itemize}
\end{itemize}

\newpage

\section{Top 10 peores epigramas}

\begin{itemize}
    \item 87. We have the mini and the microcomputer. In what semantic niche would the pico computer fall?
            \begin{itemize}
            \item \textbf{ Qué significa para ti el epigrama} 
                \newline Es una pregunta retórica que agarra los conceptos de mini y micro para extrapolar una “Pico computer”, juego de palabras feo, aunque como vamos en algunos años si sale una pico-computer.
            
                
            \item \textbf{ Por qué lo elegiste}
            \newline Le falta profundidad al epigrama y el juego de palabras fue muy poco humoristico 
        \end{itemize}
        
    \item 29. For systems, the analogue of a face-lift is to add to the control graph an edge that creates a cycle, not just an additional node.
            \begin{itemize}
            \item \textbf{ Qué significa para ti el epigrama} 
                \newline El agregar algo más complejo a un código ya realizado envés de algo más simple y que cumple el mismo objetivo.
                
            \item \textbf{ Por qué lo elegiste}
            \newline Analogía rara, sepa por qué compara el modificar un sistema con el estiramiento facial, a lo mejor en lo de estirar el programa.
        \end{itemize}
    \item 98. In computing, the mean time to failure keeps getting shorter.
            \begin{itemize}
            \item \textbf{ Qué significa para ti el epigrama} 
                \newline El equivocarse a la hora de programacion se va haciendo cada vez mas comun a medida de que el programa va creciendo. 
            
                
            \item \textbf{ Por qué lo elegiste}
            \newline El epigrama es demasiado simple y no da ninguna reflexión realmente grande, no tiene algo que haga que sea memorable, lo tuve que leer varias veces porque se me olvidaba que decía de tan simple que está.
        \end{itemize}
    \item 105. You can’t communicate complexity, only an awareness of it.
            \begin{itemize}
            \item \textbf{ Qué significa para ti el epigrama} 
                \newline Explica la dificultad que es explicar la complejidad de algo para que sea completamente entendible, solo se puede llegar a dar una noción de esta misma.
            
                
            \item \textbf{ Por qué lo elegiste}
            \newline Dio un tema relevante para la programación y lo puso en un epigrama muy simple y que no tiene algo para pensar, esto es solo una afirmación, no puedo estar en contra de ella, no me dejo reflexionando
        \end{itemize}
    \item 109. Think of it! With VLSI we can pack 100 ENIACS in 1 sq. cm.
            \begin{itemize}
            \item \textbf{ Qué significa para ti el epigrama} 
            \newline Gracias a los microchips se puede llegar a tener lo que se tenia en cuartos enteros 100 veces dentro de 1 $cm^2$
                
                
            \item \textbf{ Por qué lo elegiste}
            \newline Demasiado tecnico el epigrama, tampoco cobraban por utilizar mas palabras, la idea que da es algo sencilla pero por la forma de escribirla lo pone mas complejo de lo que deberia sin tener el conocimiento especifico
        \end{itemize}
    \item 112. Computer Science is embarrassed by the computer
            \begin{itemize}
            \item \textbf{ Qué significa para ti el epigrama} 
                \newline El avance de la tecnologia es demasiado grande para que la ciencia detras de ella se vaya actualizando de manera regular.
            
                
            \item \textbf{ Por qué lo elegiste}
            \newline La idea que el avance de la tecnologia superara de manera muy avanzada a la ciencia detras de ella es una cosa que como computologos tenemos que tener en cuenta siempre, este epigrama es pesimista a la hora de tratar la relacion de la tecnología y la ciencia
        \end{itemize}
    \item 120. Adapting old programs to fit new machines usually means adapting new machines to behave like old ones
            \begin{itemize}
            \item \textbf{ Qué significa para ti el epigrama} 
                \newline El adaptar programas a computadoras mas potentes puede hacer que no se utilice completamente toda la capacidad computacional de la maquina donde se corre el programa viejo.
            
                
            \item \textbf{ Por qué lo elegiste}
            \newline La idea de el adaptar programas viejos no utiliza completamente las capacidades de la computadora es algo real, pero tampoco es necesario que un programa que ya este comprobado su funcionamiento y la efectividad del mismo tenga que empezar a mejorar por la tecnologia en la que se va a ir adaptando.
        \end{itemize}
    \item 20. wherever there is modularity there is the potential for misunderstanding: Hiding information implies a need to check communication.
            \begin{itemize}
            \item \textbf{ Qué significa para ti el epigrama} 
            \newline El uso de la modularidad en un programa aumenta la complejidad del programa, ya que se llega a ocultar información importante.
                
            
                
            \item \textbf{ Por qué lo elegiste}
            \newline La modularidad esta chila, si se puede llegar a generar a malentendidos por la perdida de información, pero para eso se documentan los programas.
        \end{itemize}
    \item 91. The computer reminds one of Lon Chaney – it is the machine of a thousand faces.
            \begin{itemize}
            \item \textbf{ Qué significa para ti el epigrama} 
            \newline La computadora puede llegar a representar 1000 pogramas diferentes y tener usos de diferentes maneras.
                
            
                
            \item \textbf{ Por qué lo elegiste}
            \newline Referencia cultural de epoca de un actor que ni idea quién era, tuve que irme a leer un artículo de wikipedia para ver quién era el actor y a que hacía referencia el epigrama, y el chiste está feo.
        \end{itemize}
    \item 57. It is easier to change the specification to fit the program than vice versa.
            \begin{itemize}
            \item \textbf{ Qué significa para ti el epigrama} 
                \newline El cambio de los requerimientos necesarios para que un código este hecho de manera correcta son mas fáciles de cambiar que el adaptar el código para que cumplan estos requisitos 
            
                
            \item \textbf{ Por qué lo elegiste}
            \newline La idea es demasiado simple, no tiene ningún chascarrillo o que te haga pensar demasiado, comparando con los demás de la lista es el más entendible o si se ve de otra forma el más plano
        \end{itemize}
\end{itemize}

\end{document}
