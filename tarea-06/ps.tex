\documentclass{article}

\author{Gael Balderrama Dominguez}
\title{Tarea-06}
\date{Noviembre 2023}

\begin{document}
\begin{center}
    Gael Balderrama Dominguez \linebreak
    Tarea-06 \linebreak
    Noviembre 2023 \linebreak
\end{center}
Problema 4 \newline
    La llamada (bundle '("a" "b" "c") 0) es un buen uso de bundle ¿qué produce? ¿por qué? \newline
    Respuesta: \newline
    Es un buen uso, ya que deja ver como la funcion maneja los casos donde el numero en el que se van a separar las listas unitarias es menor que el numero de listas y tambien que se hace cuando la lista es 0, dando como resultado, en este caso, el resultado seria regresar la lista vacia con la definicion de bundle que manejamos

Problema 10 \newline
    Si la entrada a quicksort contiene varias repeticiones de un numero, va a regresar una lista estrictamente más corta que la entrada. Responde
    el por qué y arregla el problema. \newline
    Respuesta: \newline
    Cuando un elemento es igual al pivote, no son incluidos a la lista resultante, si no nomas se agrega el pivote, dando esto como resultado el que la lista resultante sea mas corta que la lista original ya que elimina los elementos duplicados

Problema 18 \newline
    ¿En qué casos el subproblema no es estructuralmente más pequeño que el problema original en la definición de la función de Ackermann?\newline
    Respuesta: \newline
    El caso en el que el subproblema no es mas pequeño que el problema orignal es en la tercera linea de la definicion de la funcion de Ackermann, ya que $m$ y $n$ se reducen pero tambien se utiliza $m$ sin reducir, haciendo que el problema no disminuya en cada llamada recursiva 

Problema 19 \newline
    Describe con tus propias palabras cómo funciona find-largestdivisor de gcd-structural. Responde por qué comienza desde (min n m).\newline
    Respuesta: \newline
    La funcion find-largestdivisor en gcd-structural funciona con un condicional con 3 posibles casos, uno donde el valor con el que se llama find-largestdivisor es igual a 1 entonces se regresa 1, otro donde se revisa si el valor del resudio de n y k, y el resudio de m y k, son iguales, si son iguales se regresa k, y en otro caso, se vuelve a llamar a find-largestdivisor con el valor de k-1
    El porque se eligue el minimo entre n y m es que solo es necesario comprobar los divisores menores o iguales al numero menor entre los 2 ya que si uno es menor que otro entonces es divisor del mas grande\newline 

    Problema 20 \newline 
    Describe con tus propias palabras cómo funciona find-largestdivisor de gcd-structural. Responde por qué comienza desde (min n m).\newline
    Respuesta: \newline
    El find-largestdivisor en gcd-generative recibe 2 argumentos, el $max$ y $min$ que representan los numeros mas grandes que se comparan para el Maximo comun divisor, si el min es igual a cero se devuelve max ya que seria el maximo comun divisor de los 2 numeros, si este no es el caso se llama recursivamente a find-largestdivisor con min y el residuo del max y el min

    Problema 22 \newline
    Piensa y Describe por qué no siempre es la mejor opción elegir el procedimiento más eficiente en tiempo de ejecución. Utiliza criterios que no sean el de "Eficencia"
    Respuesta: \newline
    Uno de los factores que podria llegar a ser mejor que la eficiencia del procedimiento seria la facilidad de compresion y el uso del mismo, si se tiene un procedimiento que es muy eficiente pero es dificil de comprender no se podria utilizar de manera tan efectiva o tan flexible en diferentes casos, haciendo que su uso se vea limitado a casos especificos.

\end{document}

