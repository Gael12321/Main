\usepackage{graphicx} % Required for inserting images
\documentclass{article}

\author{Gael Balderrama Dominguez}
\title{Tarea-06}
\date{Noviembre 2023}

\begin{document}
\begin{center}
    Gael Balderrama Dominguez \linebreak
    Tarea-06 \linebreak
    Noviembre 2023 \linebreak
\end{center}
Problema 4 \newline
    La llamada (bundle '("a" "b" "c") 0) es un buen uso de bundle ¿qué produce? ¿por qué? \newline
    Respuesta: \newline
    Es un buen uso, ya que deja ver como la funcion maneja los casos donde el numero en el que se van a separar las listas unitarias es menor que el numero de listas y tambien que se hace cuando la lista es 0, dando como resultado, en este caso, el resultado seria regresar la lista vacia con la definicion de bundle que manejamos

Problema 10 \newline
    Si la entrada a quicksort contiene varias repeticiones de un numero, va a regresar una lista estrictamente más corta que la entrada. Responde
    el por qué y arregla el problema. \newline
    Respuesta: \newline
    Cuando un elemento es igual al pivote, no son incluidos a la lista resultante, si no nomas se agrega el pivote, dando esto como resultado el que la lista resultante sea mas corta que la lista original ya que elimina los elementos duplicados

Problema 18 \newline
    ¿En qué casos el subproblema no es estructuralmente más pequeño que el problema original en la definición de la función de Ackermann?\newline
    Respuesta: \newline
    El caso en el que el subproblema no es mas pequeño que el problema orignal es en la tercera linea de la definicion de la funcion de Ackermann, ya que $m$ y $n$ se reducen pero tambien se utiliza $m$ sin reducir, haciendo que el problema no disminuya en cada llamada recursiva 

Problema 19 \newline
    Describe con tus propias palabras cómo funciona find-largestdivisor de gcd-structural. Responde por qué comienza desde (min n m).\newline
    Respuesta: \newline
    
Problema 20 \newline 
    Describe con tus propias palabras cómo funciona find-largestdivisor de gcd-structural. Responde por qué comienza desde (min n m).\newline
    Respuesta: \newline


\end{document}

